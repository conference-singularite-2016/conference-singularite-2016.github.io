
\documentclass[a4paper,12pt]{article}
%%%%%%%%%%%%%%%%%%%%%%%%%%%%%%%%%%%%%%%%%%%%%%%%%%%%%%%%%%%%%%%%%%%%%%%%%%%%%%%%%%%%%%%%%%%%%%%%%%%%%%%%%%%%%%%%%%%%%%%%%%%%%%%%%%%%%%%%%%%%%%%%%%%%%%%%%%%%%%%%%%%%%%%%%%%%%%%%%%%%%%%%%%%%%%%%%%%%%%%%%%%%%%%%%%%%%%%%%%%%%%%%%%%%%%%%%%%%%%%%%%%%%%%%%%%%
\usepackage{eurosym}
\usepackage{amsfonts}
\usepackage{amsmath}
\usepackage{amssymb}

 
\renewcommand*\refname{Kaynaklar}
\def\ms{\textsc{2010 AMS S\.{i}n\.{i}fland\.{i}rmas\.{i}: }}
\providecommand\key[1]{\par \textsc{Anahtar Kelimeler :}#1 \vspace{0,5cm}}
\providecommand\add[2]{\emph{{#1}} {\newline  E-Posta : \emph{#2}}}
\textheight 23cm \textwidth 13.0cm
 \oddsidemargin=1.4cm
  \evensidemargin=1.4cm
\topmargin=-0.4cm
 
\begin{document}

\title{Sets with few rational points}
\author{Georges Comte}
\date{}
\maketitle

\begin{center}
{\Large Abstract}
\end{center}
We explain how to prove that some categories of sets contain few rational points (of bounded height). 
The sets we are going to investigate are subanalytic p-adic, or even definable sets in some more general non archimedean context, 
as well as real sets slowly oscillating.      


\end{document}
