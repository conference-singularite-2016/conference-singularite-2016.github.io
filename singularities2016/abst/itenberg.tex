
\documentclass[11pt,a4paper]{article} %twocolumn
  \usepackage[francais]{babel}
	\usepackage[utf8]{inputenc}
  \usepackage[T1]{fontenc}
  \usepackage{lmodern}
  \renewcommand*\familydefault{\sfdefault} 
	\usepackage{eurosym}
	\usepackage{xspace}
  
	\usepackage{amsmath}
	\usepackage{amsthm}
	\usepackage{amssymb}
	\usepackage{sfmath}

	\usepackage{tabularx,booktabs}
%	\usepackage[a4paper, hmargin=2.5cm, vmargin=1.5cm, marginparwidth=2.5cm]{geometry}
	\usepackage{fancyhdr}
  \pagestyle{empty}
  \usepackage[empty]{fullpage}
  
  \usepackage[linesnumbered,ruled,vlined,french]{algorithm2e}
  \usepackage{listings}
  \lstdefinelanguage{PLSQL}[]{SQL}{
    morekeywords={ENUM,IF,REFERENCES,DECLARE,BEGIN,
    ELSIF,WHILE,LOOP,EXIT,FOR,REVERSE,OPEN,FETCH,CLOSE,
    IS,CURSOR,RAISE,REPLACE,PROCEDURE,CALL,RETURN,CONSTANT,
    REFERENCING,OLD,NEW,AFTER,BEFORE,EACH,ROW,MINUS},
    deletekeywords={ACTION} }[keywords,comments,strings]
  \lstset{language=PLSQL, basicstyle=\footnotesize}  

  \usepackage{bussproofs}

	\usepackage{url}
	\usepackage[a4paper]{hyperref}
	\hypersetup{
				pdfauthor   = {Romuald THION},%
				pdftitle    = {},%
				pdfsubject  = {}
		}	


  \usepackage[]{exercise}%% %noanswer answerdelayed
  \renewcommand{\ExerciseHeaderTitle}{\ExerciseTitle}
  \renewcommand{\ExerciseHeader}{{\textbf{\large
  \ExerciseHeaderDifficulty \ExerciseName~\ExerciseHeaderNB~: \ExerciseHeaderTitle
  \ExerciseHeaderOrigin\medskip}}}
  \renewcommand{\AnswerHeader}{\medskip{\textbf{\large
  \AnswerName~\ExerciseHeaderNB}\smallskip}}


\newcommand{\univers}{\ensuremath{\cal U}\xspace}
\newcommand{\valeurs}{\ensuremath{\cal D}\xspace}
\newcommand{\R}{\ensuremath{\mathbf{R}}\xspace}
\newcommand{\df}{\rightarrow}
\newcommand{\di}{\subseteq}
	


%%%%%%%%%%%%%%%%%%%%%%%%%%%%%%%%%%%%%%%%%%%%%%%%%%%%%%%%%%%%%%%%%%%%%%%%%%%%%%%%%%%%%%%%%%%%%%%%%%%%%%%%%%%%%%%
%%%%%%%%%%%%%%%%%%%%%%%%%%%%%%%%%%%%%%%%%%%%%%%%%%%%%%%%%%%%%%%%%%%%%%%%%%%%%%%%%%%%%%%%%%%%%%%%%%%%%%%%%%%%%%%


\title{\textsc{Lif{10} -- Fondements des bases de données}\\
        \emph{TD -- Introduction aux dépendances}}       
\author{Licence informatique -- printemps 2012--2013}        
\date{}%L3 Info -- printemps 2012--2013}

%%%%%%%%%%%%%%%%%%%%%%%%%%%%%%%%%%%%%%%%%%%%%%%%%%%%%%%%%%%%%%%%%%%%%%%%%%%%%%%%%%%%%%%%%%%%%%%%%%%%%%%%%%%%%%%
%%%%%%%%%%%%%%%%%%%%%%%%%%%%%%%%%%%%%%%%%%%%%%%%%%%%%%%%%%%%%%%%%%%%%%%%%%%%%%%%%%%%%%%%%%%%%%%%%%%%%%%%%%%%%%%

\begin{document}
  \maketitle
  \thispagestyle{empty}


	\begin{abstract}
	Exemples d'exercices \LaTeXe
	\end{abstract} 


%%%%%%%%%%%%%%%%%%%%%%%%%%%%%%%%%%%%%%%%%%%%%%%%%%%%%%%%%%%%%%%%%%%%%%%%%%%%%%%%%%%%%%%%%%%%%%%%%%%%%%%%%%%%%%%
%%%%%%%%%%%%%%%%%%%%%%%%%%%%%%%%%%%%%%%%%%%%%%%%%%%%%%%%%%%%%%%%%%%%%%%%%%%%%%%%%%%%%%%%%%%%%%%%%%%%%%%%%%%%%%%

\begin{Exercise}[title={vérification des dépendances en SQL}, label={TD2.3}]

Prouver que $r,s \models R[X] \di S[Y]$ \emph{si et seulement si} $|\pi_X(r) \setminus \pi_{Y}(s)|=0$, en déduire  une requête SQL qui permet de tester la satisfaction d'une dépendance d'inclusion.

\end{Exercise}

\begin{Answer}[ref={TD2.3}]

\begin{lstlisting}
SELECT COUNT(*)
FROM  (SELECT DISTINCT X FROM r
       MINUS
       SELECT DISTINCT Y FROM s)

\end{lstlisting}
\end{Answer}





\begin{Exercise}[title={}, label={TD3.1}]


\begin{minipage}{0.45\textwidth}
\begin{center}
    \begin{prooftree}
    \AxiomC{$Y \subseteq X$}
    \RightLabel{$\sigma_R$ (réflexivité)}
    \UnaryInfC{$X \df Y $}
    \end{prooftree}
    \begin{prooftree}
    \AxiomC{$X \df Y$}
    \RightLabel{$\sigma_A$ (augmentation)}
    \UnaryInfC{$WX \df WY $}
    \end{prooftree}
    \begin{prooftree}
    \AxiomC{$X \df Y$}
    \AxiomC{$Y \df Z$}
    \RightLabel{$\sigma_T$ (transitivité)}
    \BinaryInfC{$X \df Z$}
    \end{prooftree}
\end{center}
\end{minipage}
\begin{minipage}{0.45\textwidth}
\begin{center}
    \begin{prooftree}
    \AxiomC{$X \df Y$}
    \AxiomC{$X \df Z$}
    \RightLabel{$\sigma_C$ (composition)}
    \BinaryInfC{$X \df YZ$}
    \end{prooftree}
    \begin{prooftree}
    \AxiomC{$X \df YZ$}
    \RightLabel{$\sigma_D$ (décomposition)}
    \UnaryInfC{$X \df Y$}
    \end{prooftree}
    \begin{prooftree}
    \AxiomC{$X \df Y$}
    \AxiomC{$WY \df Z$}
    \RightLabel{$\sigma_P$ (pseudo-transitivité)}
    \BinaryInfC{$WX \df Z$}
    \end{prooftree}
\end{center}
\end{minipage}

Soit $\Sigma$ l'ensemble des dépendances suivantes

\begin{center}
\begin{minipage}[t]{.32\linewidth}
    $BC\df A$\\
    $AC\df B$\\
    $AE\df C$
  \end{minipage}
  \begin{minipage}[t]{.32\linewidth}
    $D\df BE$ \\ 
    $B\df DE$\\    
    $C\df E$
  \end{minipage}
\end{center}

\Question 
\begin{enumerate}
\item $AD\df C$
\item $CD\df A$
\end{enumerate}


\Question Même question en calculant la fermeture des parties gauches à partir de l'algorithme de fermeture.
\end{Exercise}



\begin{Answer}[ref={TD3.1}]

  \Question

  \begin{enumerate}
    \item 
  \begin{prooftree}
      \AxiomC{$D \df BE$}
      \RightLabel{aug.}
      \UnaryInfC{$AD \df ABE$}
      \AxiomC{$AE \di ABE$}
      \RightLabel{refl.}
      \UnaryInfC{$ABE \df AE$}
      \RightLabel{trans.}
      \BinaryInfC{$AD \df AE$}
      \AxiomC{$AE \df C$}
      \RightLabel{trans.}
      \BinaryInfC{$AD \df C$}
      \end{prooftree}

  \item 
      \begin{prooftree}
      \AxiomC{$D \df BE$}
      \RightLabel{decomp.}
      \UnaryInfC{$D \df B$}
      \RightLabel{aug.}
      \UnaryInfC{$CD \df BC$}
      \AxiomC{$BC \df A$}
      \RightLabel{trans.}
      \BinaryInfC{$CD \df A$}
      \end{prooftree}
  \end{enumerate}

  \Question
  \begin{enumerate}
    \item Pour  $AD \df C$, les étapes successives de l'algorithme sont les suivantes :
    \begin{enumerate}
      \item $closure = AD$
      \item $closure = ABDE$ en utilisant $D\df BE$ 
      \item $closure = ABCDE$ en utilisant $AE\df C$
      \item comme $C \subseteq AD^+$, on en déduit $AD \df C$ par correction de l'algorithme
    \end{enumerate}
      \item Pour  $CD \df A$, les étapes successives de l'algorithme sont les suivantes :
    \begin{enumerate}
      \item $closure = CD$
      \item $closure = BCDE$ en utilisant $D\df BE$ 
      \item $closure = ABCDE$ en utilisant $BC\df A$
      \item comme $A \subseteq CD^+$, on en déduit $CD \df A$ par correction de l'algorithme
    \end{enumerate}
  \end{enumerate}
\end{Answer}




\begin{algorithm}
\KwData{ $\Sigma$ un ensemble de DF, $X$ un ensemble d'attributs.}
\KwResult{ $X^+$, la fermeture de $X$ par $\Sigma$.}
$unused := \Sigma$\\
$closure := X$\\
\Repeat{$closure' = closure$}
	{ $closure':= closure$\;
	\If{$W\df Z \in unused \text{ and } W \subseteq closure$}
		{
		$unused := unused - \{W\df Z \}$\;
		$closure:= closure \cup Z$\;
		}
	}	 
\Return $closure$\;
\label{algo}
\caption{fermeture d'un ensemble d'attributs par des dépendances fonctionnelles}
\end{algorithm}

%%%%%%%%%%%%%%%%%%%%%%%%%%%%%%%%%%%%%%%%%%%%%%%%%%%%%%%%%%%%%%%%%%%%%%%%%%%%%%%%%%%%%%%%%%%%%%%%%%%%%%%%%%%%%%%
%%%%%%%%%%%%%%%%%%%%%%%%%%%%%%%%%%%%%%%%%%%%%%%%%%%%%%%%%%%%%%%%%%%%%%%%%%%%%%%%%%%%%%%%%%%%%%%%%%%%%%%%%%%%%%%

%A (dé)commenter lors de l'utilisation du paramètre answerdelayed du package {exercise} pour avoir les
%corrections séparées
 
%\clearpage
%\section{Corrections}
%\shipoutAnswer


%%%%%%%%%%%%%%%%%%%%%%%%%%%%%%%%%%%%%%%%%%%%%%%%%%%%%%%%%%%%%%%%%%%%%%%%%%%%%%%%%%%%%%%%%%%%%%%%%%%%%%%%%%%%%%%
%%%%%%%%%%%%%%%%%%%%%%%%%%%%%%%%%%%%%%%%%%%%%%%%%%%%%%%%%%%%%%%%%%%%%%%%%%%%%%%%%%%%%%%%%%%%%%%%%%%%%%%%%%%%%%%
%Bibliographic references

%	\bibliographystyle{abbrv}
%	\bibliography{biblio}

\end{document}