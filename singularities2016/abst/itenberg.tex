
\documentclass[a4paper,12pt]{article}
%%%%%%%%%%%%%%%%%%%%%%%%%%%%%%%%%%%%%%%%%%%%%%%%%%%%%%%%%%%%%%%%%%%%%%%%%%%%%%%%%%%%%%%%%%%%%%%%%%%%%%%%%%%%%%%%%%%%%%%%%%%%%%%%%%%%%%%%%%%%%%%%%%%%%%%%%%%%%%%%%%%%%%%%%%%%%%%%%%%%%%%%%%%%%%%%%%%%%%%%%%%%%%%%%%%%%%%%%%%%%%%%%%%%%%%%%%%%%%%%%%%%%%%%%%%%
\usepackage{eurosym}
\usepackage{amsfonts}
\usepackage{amsmath}
\usepackage{amssymb}

 
\renewcommand*\refname{Kaynaklar}
\def\ms{\textsc{2010 AMS S\.{i}n\.{i}fland\.{i}rmas\.{i}: }}
\providecommand\key[1]{\par \textsc{Anahtar Kelimeler :}#1 \vspace{0,5cm}}
\providecommand\add[2]{\emph{{#1}} {\newline  E-Posta : \emph{#2}}}
\textheight 23cm \textwidth 13.0cm
 \oddsidemargin=1.4cm
  \evensidemargin=1.4cm
\topmargin=-0.4cm
 
\begin{document}

\title{Relative enumerative invariants of real rational surfaces}
\author{Illia Itenberg}
\date{}
\maketitle

\begin{center}
{\Large Abstract}
\end{center}
The purpose of the talk is to present real analogs of relative Gromov-Witten invariants in several situations.
For example, for real del Pezzo surfaces with a real (-2)-curve, we suggest, under some assumptions,
an invariant signed count of real rational curves 
that belong to a given divisor class and are tangent to the (-2)-curve at each intersection point;
the resulting number does not depend neither on the point constrains,
nor on deformation of the surface preserving the real structure and the (-2)-curve.
(Joint work with V. Kharlamov and E. Shustin.)


\end{document}
